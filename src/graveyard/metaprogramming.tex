w\section{Introduction to Meta Programming}

(relate to 2ltt first)

% what is metaprogramming
% racket 

We take a side step to look at a couple of concrete examples in meta-programming, which will serve as intuition for understanding later sections.

With tools such as macros in Rust or templates in Haskell, one can write metaprograms that only needs to provide a small amount of code (like \lstinline[mathescape]{deriving Show}) to get relatively complex behaviour in return. Amazingly, these two languages' meta-programming systems are complex and powerful enough that avoid errors a string replacement macro system in C can introduce.

Now, suppose you want to introduce such features (what features) into a different language. One way of implementing this might be to dive deep (wdym) into the compiler and specify the semantics of some macros. However, another approach might be to translate (these techniques) into the language of the core language. 

ML is a functional programming language. One famous contribution of the language is the proof that ``a well-typed ML program does not cause runtime type errors''. MetaML is an extension of the language, introducing new types that are representations of other native ML types. 

MetaML introduces two operators, 
\begin{itemize}
    \item \textit{quote} (\lstinline[mathescape]{<->}): Used to construct a representation of value
    \item \textit{splice} (\lstinline[mathescape]{$\spl$ -}): Evaluate a representation into a value
\end{itemize} 

The following is a line of code in MetaML:

\begin{lstlisting}
    let val x = <1 + 4> in <72 + (~x)>
\end{lstlisting}

\lstinline[mathescape]{val x = <1 + 4>} is defining a local variable \texttt{x} that is a representation of \lstinline[mathescape]{1 + 4}.  It is important to know that \texttt{x} is not an integer, only a representation as indicated by the quote operator. If it were to be separated, \texttt{x} would have the type \texttt{<int>}, the angle bracket in a type indicates that the type is a representation of the inner type. Recognizing the meaning and the type of \texttt{x}, the $\spl$ in \texttt{($\spl$x)} evaluates the value that \texttt{x} represents, and injects the integer value into \lstinline[mathescape]{72 + -}. The expression \lstinline[mathescape]{72 + -} is then once again returned as a representation of an integer by the \texttt{<->} surrounding it.  

Delaying evaluation by treating some expressions as the representation of values is at the core of MetaML. We encourage the reader to explore MetaML. However, our example should serve as an entry point to the main topic of Two-Level Type Theory. 

We can think of types with and without quotes to be in different worlds, linked only through the correct placement of quotes and splices throughout the implementation of the type. This in essence is the study of Two-Level Type Theory, which the rest of the paper shall explore.


