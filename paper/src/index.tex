\documentclass{article}
\usepackage{indentfirst} % indent first paragraph
\usepackage[style=alphabetic,sorting=ynt,backend=bibtex]{biblatex}
\usepackage{amsmath,amssymb,amsfonts,amsthm}
\usepackage{xcolor}
\usepackage[colorlinks]{hyperref}
\usepackage[shortlabels]{enumitem}
\usepackage[margin=0.75in]{geometry}
\usepackage{mathpartir}
\usepackage{scalerel}

\RequirePackage[T1]{fontenc}
\RequirePackage[tt=false, type1=true]{libertine}
\RequirePackage[varqu]{zi4}
\RequirePackage[libertine]{newtxmath}

\addbibresource{refs.bib}
\addbibresource{short.bib}
%\setmonofont{iosevka-custom-regular.ttf}
%\setmainfont{Roboto-Regular.ttf}

\definecolor{red}{HTML}{C63A44}
\definecolor{blue}{HTML}{0000aa}
\definecolor{green}{HTML}{008000}



\renewcommand{\labelitemii}{$\labelitemfont\bullet$}


\newcommand\lt<
\newcommand\gt>
\providecommand\tightlist{\setlength{\itemsep}{0in}\setlength{\parskip}{0in}}
\newcommand\sube\subseteq
\newcommand\sub\subset

\newcommand{\Lift}{{\Uparrow}}
\newcommand{\spl}{{\sim}}
\newcommand{\qut}[1]{\langle #1\rangle}

\newcommand{\msf}[1]{\mathsf{#1}}
\newcommand{\mbb}[1]{\mathbb{#1}}
\newcommand{\mbf}[1]{\mathbf{#1}}
\newcommand{\bs}[1]{\boldsymbol{#1}}
\newcommand{\wh}[1]{\widehat{#1}}
\newcommand{\ext}{\triangleright}
\newcommand{\Code}{\msf{Code}}
\newcommand{\El}{\msf{El}}
\newcommand{\lam}{\msf{lam}}
\newcommand{\app}{\msf{app}}
\newcommand{\NatElim}{\msf{NatElim}}
\newcommand{\y}{\msf{y}}

\newcommand{\U}{\msf{U}}
\newcommand{\Con}{\msf{Con}}
\newcommand{\Sub}{\msf{Sub}}
\newcommand{\Ty}{\msf{Ty}}
\newcommand{\Tm}{\msf{Tm}}

\newcommand{\Set}{\mathsf{Set}}
\newcommand{\Prop}{\mathsf{Prop}}
\newcommand{\Rep}{\msf{Rep}}
\newcommand{\blank}{{\mathord{\hspace{1pt}\text{--}\hspace{1pt}}}}
\newcommand{\emb}[1]{\ulcorner#1\urcorner}


\newcommand{\refl}{\msf{refl}}
\newcommand{\Bool}{\msf{Bool}}
\newcommand{\true}{\msf{true}}
\newcommand{\false}{\msf{false}}
\newcommand{\True}{\msf{True}}
\newcommand{\False}{\msf{False}}
\newcommand{\List}{\msf{List}}
\newcommand{\nil}{\msf{nil}}
\newcommand{\cons}{\msf{cons}}
\newcommand{\Nat}{\msf{Nat}}
\newcommand{\zero}{\msf{zero}}
\newcommand{\suc}{\msf{suc}}
\renewcommand{\tt}{\msf{tt}}
\newcommand{\fst}{\msf{fst}}
\newcommand{\snd}{\msf{snd}}
\newcommand{\mylet}{\msf{let}}
\newcommand{\emptycon}{\scaleobj{.75}\bullet}
\newcommand{\id}{\msf{id}}

\newcommand{\p}{\mathsf{p}}
\newcommand{\q}{\mathsf{q}}


\theoremstyle{plain}
\newtheorem{theorem}{Theorem}[subsection]

\theoremstyle{definition}
\newtheorem{definition}{Definition}[subsection]
\newtheorem{example}{Example}[subsection]
\newtheorem{property}{Property}[subsection]
\newtheorem{corollary}{Corollary}[theorem]
\newtheorem{lemma}{Lemma}[theorem]

\theoremstyle{remark}
\newtheorem{notation}{Notation}[subsection]
\newtheorem{remark}{Remark}[subsection]



\newcommand\lamb[1]{\lambda\foreach\a in{#1}{\,\a}.\;}
\newcommand\App[1]{\foreach\a in{#1}{\a\,}}



\title{Introduction to Staged Compilation and Two-Level Type Theory}
\date{}
\author{Yulong Liu\and Youzhang Sun}
\begin{document}

\maketitle

\section{Introduction to Staged Compilation}
The purpose of staged compilation is to write expressive metaprograms that generate code with the guarantees that the generated code is well-formed. To justify the well-formedness of the code output, the model of two-level type theory (2LTT) \cite{2ltt} is employed as a formal typing system for staged compilation. While languages such as MetaML \cite{10.1145/258994.259019} supports metaprogramming, 2LTT additionally supports dependent types. 
% Therefore, 2LTT can ensure that each metaprogram generates a well-formed output code and provide the additional expressiveness from dependent types.
Therefore, 2LTT contributes to the field of programming language and type theory by introducing dependent types to metaprogramming.

% Staged compilation is the process of compiling a program from one language to another through a staging algorithm. For a compiled language, the compiler translates the source code written in the compile time language to a program written in the runtime language, often as the language of CPU instructions. In languages that support code generation, there also exist two stages between programs with code-generating annotations (e.g. macros, generics, function inlining) to an output code without these annotations (e.g. by substituting the inlined function inside the body of the caller function). In the case of annotations as additional syntax for the compile time language, the compiler performs staged compilation by staging these annotations away; in other words, it substitutes these annotations with code in the runtime language.

% Examples of code generation with staged compilation include macros in C and generics in Rust. The preprocessor of C takes the source code and substitutes each usage of macros with their corresponding replacement code, thus performing code generation. The Rust compiler behaves similarly with respect to programs that uses generic functions: it makes copies of each generic function with the type parameters substituted by concrete types.

In this paper, we focus on metaprogramming with two stages. We index these stages as stage $0$ and stage $1$. Each stage has a language which we will further formalise into a type system. We use the term \emph{staging algorithm} to refer the process of transforming (i.e. staging) a metaprogram to a program that only uses the stage $0$ language. A metaprogram is a term with stage $0$ type but uses type/terms from the stage $1$ language through staging annotations. To explain how a metaprogram can use stage $1$ language despite being at stage $0$, we first describe the interaction between these two stages.



\subsection{Interaction between Stages}
\input{01-staged/1-interact}

\subsection{Example of A Staged Program}\label{example-staged}
\input{01-staged/2-example}

\subsection{Soundness of Staging}
\input{01-staged/3-correct}

\section{Introduction to Two-Level Type Theory}

Two-level type theory (2LTT), as the name suggests, extends type theory to two levels, which in our case, are the stages $0$ and $1$. 2LTT is useful since we can extend the language of stage $0$ as the stage $1$ language to derive useful properties that are not expressable within stage $0$; namely, the soundness of a staging algorithm.

To apply 2LTT in staged programming, we consider the languages used in the two stages as separate type systems. This separation provides support for a wide range of languages even with completely different syntax. For instance, staged compilation with 2LTT is applicable to domain specific languages whose implementations are in different languages (e.g. staging LINQ expressions to C\# method calls \cite{linq}). % 2LTT also restricts the interaction between these two theories through the three special operations: quoting, lifting, and splicing.

Since 2LTT formalises our metaprograms and output code into type theories, we provide an example of a 2LTT model that consists of universe hierarchies, type formers with dependent types, and formers/eliminators for natural numbers. We will also present a portion of the inference rules for the purpose of formalising the staging operations and working with natural numbers. Next, we will revisit the $\mathsf{double}_0$ metaprogram for type-checking and applying the staging algorithm with these inference rules. We will then round off this paper with some discussions and conclusion.



\subsection{Universes and Type Formers}
\input{02-2ltt/1-universe}


\section{Inference Rules of 2LTT}

In this section, we analyze the inference rules of 2LTT. As 2LTT is an extension of Martin-Löf type theory, many concepts involve to dependent functions. Therefore, to denote the type of a dependent function, we use the notation $(x:A)\to B$ where the term $x$ may occur in the type $B$. We also use the alternative notation $(x:A)\to B\,x$, which clarifies that $B$ is a type dependent on $x$.

 
\subsection{Judgments}
\input{03-infer/1-judge.tex}

\subsection{Familiar Inference Rules in the Context of 2LTT}
\input{03-infer/2-rules}

\input{03-infer/4-nat}


\subsection{Lifting, Quoting, and Splicing}
\input{03-infer/3-lift}



\section{Applying Judgments and Inference Rules}
In this section, we provide examples that employ the inference rules as presented in the previous section. First, we apply the inference rules to derive the type of $\mathsf{mul}_1$ under the assumption that $\mathsf{add}_0$ has type $\mathsf{Nat}_0\to\mathsf{Nat}_0\to\mathsf{Nat}_0$. The type of $\mathsf{add}_0$ is derived in the same manner since $\mathsf{mul}_1$ and $\mathsf{add}_0$ are both defined with $\mathsf{NatElim}$. Second, we use the inference rules for lifting to prove the isomorphism between lifted function types and function types where the domain and codomain are lifted. This isomorphism property is useful as it can optimise the implementation of the 2LTT model.


\subsection{Type Derivation for $\mathsf{mul}_1$}

\input{04-apply/1-mul}

\subsection{Staging $\mathsf{double_0}$ Formally}

\input{04-apply/2-double.tex}

\section{Discussion}
We have completed our main objective of applying and type-checking our metaprogram $\mathsf{double}_0$. Now we will discuss some properties of 2LTT that can serve as direction for further exploration.

\subsection{Isomorphism Property of Lifting and Quoting}
\input{05-iso}




\section{Conclusion}
We introduced staged compilation, more specifically, two-stage compilation as a useful way to write metaprograms that generate code with safety. With the special operations for moving between stages (namely lifting, quoting, and splicing) integrated to the typing rules, the 2LTT model provides not only guarantees the well-formedness of the generated output, but also support for dependent types on both stages. While this paper only covered the typing rules for type-checking a metaprogram, the staging algorithm (also known as the substitution calculus) can alse be formalised into typing rules through categorical logic (also known as abstract nonsense logic).

% Should attempt to explore the challenge of allowing type formers to form types using types from different stages. It is my suspicion that the challenge would be considering which universe would the produced type be in

\printbibliography

\end{document}

